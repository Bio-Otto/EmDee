\documentclass[11pt]{article}

\usepackage{amsmath}
\usepackage{amssymb}

\newcommand{\mt}[1]{\boldsymbol{\mathbf{#1}}}           % matrix symbol
\newcommand{\vt}[1]{\boldsymbol{\mathbf{#1}}}           % vector symbol
\newcommand{\tr}[1]{#1^\text{t}}                        % transposition
\newcommand{\diff}[2]{\frac{\partial #1}{\partial #2}}  % partial derivative
\newcommand{\im}{\hat{\i}}                              % imaginary number

\begin{document}

\title{EmDee: A Molecular Dynamics Laboratory}

\author{Charlles R. A. Abreu}
%\email{abreu@eq.ufrj.br}
%\affiliation{Chemical Engineering Department, Escola de Quimica, Universidade Federal do Rio de Janeiro, Rio de Janeiro, RJ 21941-909, Brazil}

\date{\today}

\maketitle

\section{Projections and Vector Differentiation}

Consider a column vector $\vt x \in \mathcal{R}^n$, whose norm is $x = \sqrt{\tr{\vt x}{\vt x}}$. A unit vector in the direction of $\vt x$ is given by $\vt{\hat x} = {\vt x}/x$. When premultiplied by an arbitrary vector in $\mathcal{R}^n$, the matrix obtained by the product $\vt{\hat x}\tr{\vt{\hat x}}$ provides the component of such vector in the direction of $\vt x$, while $\mt I - \vt{\hat x}\tr{\vt{\hat x}}$ provides its projection onto the plane orthogonal to $\vt x$. Such as any projection matrix, both $\vt{\hat x}\tr{\vt{\hat x}}$ and $\mt I - \vt{\hat x}\tr{\vt{\hat x}}$ are idempotent.

Here we deal with column vectors and employ the so-called Jacobian formulation (also known as numerator layout) for derivatives. In this case, differentiation of a vector-valued function $\vt y \in \mathcal{R}^m$ with respect to a vector variable $\vt x \in \mathcal{R}^n$ results in a matrix $\mt J \in \mathcal{R}^{m \times n}$ so that $J_{ij} = \partial y_i / \partial x_j$. With this formulation, the total differential of a generic function $\vt z = {\vt z}(\vt x,\vt y)$ is given by 
\[
d{\vt z} = \diff{\vt z}{\vt x} d{\vt x} + \diff{\vt z}{\vt y} d{\vt y},
\]
which is valid even if $\vt x$, $\vt y$ or $\vt z$ is a scalar. From this expression, many useful differentiation rules involving vectors can be derived. For instance,
\begin{equation}
\label{eq:prod_rule_1}
d(\tr{\vt x}{\vt y}) = \tr{\vt x}d{\vt y} + \tr{\vt y}d{\vt x}
\end{equation}
and
\begin{equation}
\label{eq:prod_rule_2}
d({\vt x}y) = {\vt x}dy + y d{\vt x}.
\end{equation}

A special case of Eq.~\ref{eq:prod_rule_1} is $d(\tr{\vt x}{\vt x}) = 2\tr{\vt x}d{\vt x}$, which can be used to obtain the differential $dx = d(\sqrt{\tr{\vt x}{\vt x}}) = \frac{1}{2}(\tr{\vt x}\vt x)^{-1/2}d(\tr{\vt x}\vt x)$, whose final form is
\begin{equation}
\label{eq:norm_rule}
dx = \frac{\tr{\vt x}d{\vt x}}{x}.
\end{equation}

Similarly, we have $d({\vt x}x^{-1}) = -{\vt x}x^{-2}dx + x^{-1}d\vt x$ as a special case of Eq.~\ref{eq:prod_rule_2}. This can be rewritten as
\begin{equation}
\label{eq:unit_vector_rule}
d\hat{\vt x} = \frac{\mt I - \hat{\vt x}\tr{\hat{\vt x}}}{x} d\vt x,
\end{equation}
which shows us that a differential change in a vector $\vt x$ causes an orthogonal differential change in the unit vector it generates. Next, we consider the differential form of a projection like $\hat{\vt x} \tr{\hat{\vt x}} \vt y$. By applying Eq.~\ref{eq:prod_rule_2}, followed by Eq.~\ref{eq:prod_rule_1}, it turns out that
\begin{equation}
\label{eq:projection_rule}
d(\hat{\vt x} \tr{\hat{\vt x}} \vt y) = \hat{\vt x}\tr{\hat{\vt x}} d\vt y + (\hat{\vt x}\tr{\vt y} + \tr{\hat{\vt x}} \vt y \mt I) d\hat{\vt x}.
\end{equation}

Finally, consider a pair of matrix-valued functions with vector arguments, $A$ and $B$, defined so that $A(\vt x)\vt y = \vt z = B(\vt y)\vt x$. In this case, it follows from Eq.~\ref{eq:prod_rule_1} that
\begin{equation}
\label{eq:interchange_rule}
d\vt z = A(\vt x)d\vt y + B(\vt y)d\vt x.
\end{equation}

\section{Torsional Forces}

In order to calculate the torsional forces, we begin by defining an orthonormal, right-handed basis composed by the vectors
\begin{subequations}
\begin{align}
\hat{\vt z} &= \dfrac{{\vt r}_{kj}}{r_{kj}}, \\
\hat{\vt x} &= \dfrac{\vt x}{x}, \, \text{where} \, \vt x = (\mt I - \hat{\vt z}\tr{\hat{\vt z}}){\vt r}_{ij}, \, \text{and} \label{eq:x_definition} \\
\hat{\vt y} &= \hat{\vt z} \times \hat{\vt x}.
\end{align}
\end{subequations}

This serves as an internal coordinate system for the dihedral. Since it is an orthonormal basis for $\mathcal{R}^3$, it follows that $\tr{\hat{\vt x}}\hat{\vt y} = \tr{\hat{\vt x}}\hat{\vt z} = \tr{\hat{\vt y}}\hat{\vt z} = 0$ and that $\hat{\vt x}\tr{\hat{\vt x}} + \hat{\vt y}\tr{\hat{\vt y}} + \hat{\vt z}\tr{\hat{\vt z}} = {\mt I}$. The dihedral angle can be calculated using the arctangent function with two arguments,
\begin{subequations}
\begin{equation}
\phi = \arctan(b,a),
\end{equation}
where
\begin{align}
a &= \tr{\hat{\vt x}}{\vt r}_{lk} \quad \text{and}\\
b &= \tr{\hat{\vt y}}{\vt r}_{lk}.
\end{align}
\end{subequations}

The differential of $\phi$ is given by
\[
d\phi = \frac{a db - b da}{a^2 + b^2} = \frac{\tr{(a\hat{\vt y}-b\hat{\vt x})}d{\vt r}_{lk} + \tr{{\vt r}_{lk}}(a d\hat{\vt y} - b d\hat{\vt x})}{a^2 + b^2}.
\]

Note that $\hat{\vt y}$ is obtained from $\hat{\vt z}$ and $\hat{\vt x}$ by a cross product, which can be expressed in matrix form as $\hat{\vt y} = S(\hat{\vt z})\hat{\vt x}$, where $S(\hat{\vt z})$ is an antisymmetric matrix. Since $\hat{\vt z} \times \hat{\vt x} = - \hat{\vt x} \times \hat{\vt z}$, then $S(\hat{\vt z})\hat{\vt x} = - S(\hat{\vt x})\hat{\vt z}$ and, from Eq.~\ref{eq:interchange_rule}, it follows that $d\hat{\vt y} = S(\hat{\vt z})d\hat{\vt x} - S(\hat{\vt x})d\hat{\vt z}$. Therefore,
\[
\tr{{\vt r}_{lk}}d\hat{\vt y} = \tr{{\vt r}_{lk}}S(\hat{\vt z})d\hat{\vt x} - \tr{{\vt r}_{lk}}S(\hat{\vt x})d\hat{\vt z} = \tr{({\vt r}_{lk} \times \hat{\vt z})}d\hat{\vt x} - \tr{({\vt r}_{lk} \times \hat{\vt x})}d\hat{\vt z}.
\]

Substituting this result into $d\phi$ then yields
\[
(a^2 + b^2)d\phi = \tr{\left(a\hat{\vt y} - b\hat{\vt x}\right)}d{\vt r}_{lk}
+ \tr{\left(a{\vt r}_{lk} \times \hat{\vt z} - b{\vt r}_{lk}\right)}d\hat{\vt x}
- \tr{\left(a{\vt r}_{lk} \times \hat{\vt x}\right)}d\hat{\vt z}.
\]

From Eqs.~\ref{eq:unit_vector_rule}, \ref{eq:projection_rule}, and \ref{eq:x_definition}, it follows that
\[
d\hat{\vt x} = \frac{\mt I - \hat{\vt x}\tr{\hat{\vt x}}}{x} \left[ (\mt I - \hat{\vt z}\tr{\hat{\vt z}})d{\vt r}_{ij} - (\hat{\vt z}\tr{{\vt r}_{ij}} + \tr{\hat{\vt z}} {\vt r}_{ij} \mt I) d\hat{\vt z} \right]
\]

Using the identities $(\mt I - \hat{\vt x}\tr{\hat{\vt x}})(\mt I - \hat{\vt z}\tr{\hat{\vt z}}) = \hat{\vt y}\tr{\hat{\vt y}}$ and $(\mt I - \hat{\vt x}\tr{\hat{\vt x}})\hat{\vt z} = \hat{\vt z}$, and the fact that $\tr{\hat{\vt z}} {\vt r}_{ij}$ is a scalar, we have
\[
d\hat{\vt x} = \frac{\hat{\vt y}\tr{\hat{\vt y}}}{x}d{\vt r}_{ij} - \frac{\hat{\vt z}\tr{{\vt r}_{ij}} + \tr{\hat{\vt z}} {\vt r}_{ij} (\mt I - \hat{\vt x}\tr{\hat{\vt x}}) }{x}d\hat{\vt z}
\]

Now, for simplification, we define
\begin{subequations}
\begin{align}
&\vt u = \frac{a ({\vt r}_{lk} \times \hat{\vt z}) - b {\vt r}_{lk}}{x}, \\
&\vt v = \frac{a ({\vt r}_{lk} \times \hat{\vt x}) + (\tr{\hat{\vt z}}\vt u) {\vt r}_{ij}}{r_{kj}}, \; \text{and} \\
&\vt w = {\vt v} + \frac{\tr{\hat{\vt z}}{\vt r}_{ij}}{r_{kj}} {\vt u}
\end{align}
\end{subequations}

Next, substituting $\vt u$ and $\vt v$ and observing that $d\hat{\vt z} = (\mt I - \hat{\vt z}\tr{\hat{\vt z}}) d{\vt r}_{kj}/r_{kj}$, we obtain
\[
(a^2 + b^2)d\phi = \tr{\left(a\hat{\vt y} - b\hat{\vt x}\right)}d{\vt r}_{lk}
+ \tr{\vt u}\hat{\vt y}\tr{\hat{\vt y}}d{\vt r}_{ij} - \frac{ r_{kj} \tr{\vt v} + \tr{\hat{\vt z}} {\vt r}_{ij} \tr{\vt u}(\mt I - \hat{\vt x}\tr{\hat{\vt x}}) }{r_{kj}} (\mt I - \hat{\vt z}\tr{\hat{\vt z}}) d{\vt r}_{kj}.
\]

To conclude, the identities $(\mt I - \hat{\vt x}\tr{\hat{\vt x}})(\mt I - \hat{\vt z}\tr{\hat{\vt z}}) = \hat{\vt y}\tr{\hat{\vt y}}$ and $\mt I - \hat{\vt z}\tr{\hat{\vt z}} = \hat{\vt x}\tr{\hat{\vt x}} + \hat{\vt y}\tr{\hat{\vt y}}$, together with the definition of $\vt w$, yield
\[
(a^2 + b^2)d\phi = \tr{\left(a\hat{\vt y} - b\hat{\vt x}\right)}d{\vt r}_{lk}
+ \tr{\vt u}\hat{\vt y}\tr{\hat{\vt y}}d{\vt r}_{ij} - (\tr{\vt v}\hat{\vt x}\tr{\hat{\vt x}} + \tr{\vt w}\hat{\vt y}\tr{\hat{\vt y}}) d{\vt r}_{kj}.
\]

For a given torsional potential $U(\phi)$, the force exerted on an atom $m$ is obtained by
\[
\vt{F}_m = -\diff{U}{\phi} \tr{ \left( \diff{\phi}{\vt{r}_m} \right) }
\]

Therefore,
\begin{subequations}
\begin{align}
&\vt{F}_i = -\diff{U}{\phi} \frac{(\tr{\vt u}\hat{\vt y})\hat{\vt y}}{a^2 + b^2} \\
&\vt{F}_j = -\diff{U}{\phi} \frac{ (\tr{\vt v}\hat{\vt x})\hat{\vt x} + (\tr{\vt w}\hat{\vt y} - \tr{\vt u}\hat{\vt y}) \hat{\vt y}}{a^2 + b^2} \\
&\vt{F}_k = -\diff{U}{\phi} \frac{ (b - \tr{\vt v}\hat{\vt x})\hat{\vt x} - (a + \tr{\vt w}\hat{\vt y}) \hat{\vt y}}{a^2 + b^2} \\
&\vt{F}_l = -\diff{U}{\phi} \frac{a\hat{\vt y} - b\hat{\vt x}}{a^2 + b^2}
\end{align}
\end{subequations}

\section{Ewald Sum with Type-Pair-Specific Scaling}

\subsection{Real Part}

\begin{equation*}
U_\text{short} = \frac{k_e}{\epsilon} \sum_{i=1}^{N-1} \sum_{j=i+1}^N q_i q_j \frac{\text{erfc}(\alpha r_{ij})}{r_{ij}}
\end{equation*}

\subsection{Fourier Part}

Conventionally, for a simulation box with periodic boundary conditions, the long-range electrostatic part of the potential energy is written as
\begin{equation*}
U_\text{long} = \frac{2\pi k_e}{\epsilon V} \sum_{\vt n \neq \vt 0} \frac{e^{-\frac{k^2}{4\alpha^2}}}{k^2} \sum_{i=1}^N \sum_{j=1}^N q_i q_j e^{\im \tr{\vt k}({\vt r}_i - {\vt r}_j)},
\end{equation*}
where $k_e$ is Coulomb's constant ($k_e = \frac{1}{4\pi\epsilon_0}$), $\epsilon$ is a dielectric constant, $V$ is the box volume, $\vt n \in \mathbb Z^3$ is an integer lattice vector, $\vt k = 2\pi \vt L^{-1}{\vt n}$ is a reciprocal space vector, $k = \|\vt k\|$ is the norm of $\vt k$, and $\im$ is the imaginary unit (that is, $\im^2 = -1$). The summations over indices $i$ and $j$ run for all atoms, which allows us to split them as follows,
\begin{equation*}
U_\text{long} = \frac{2\pi k_e}{\epsilon V}\sum_{\vt n \neq \vt 0} \frac{e^{-\frac{k^2}{4\alpha^2}}}{k^2} \Bigg(\sum_{i=1}^N q_i e^{\im \tr{\vt k}{\vt r}_i} \Bigg) \Bigg(\sum_{j=1}^N q_j e^{-\im \tr{\vt k}{\vt r}_j}\Bigg).
\end{equation*}

Defined as $S(\vt k) = \sum_i q_i e^{\im \tr{\vt k}{\vt r}_i}$, the structure factor of the system charges can be identified in the equation above, as well as its complex conjugate ${\overline S}(\vt k)$. Therefore, we can write
\begin{equation*}
U_\text{long} = \frac{2\pi k_e}{\epsilon V}\sum_{\vt n \neq \vt 0} \frac{e^{-\frac{k^2}{4\alpha^2}}}{k^2} |S(\vt k)|^2.
\end{equation*}

Now consider that the electrostatic interactions between any atom $i$ of type $I$ with any atom $j$ of type $J$ are scaled by a factor $\lambda_{IJ} = \lambda_{JI} = k_e/\epsilon_{IJ}$. In this case, one can alternatively express the long-range potential per unit box as
\begin{equation*}
U_\text{long} = \frac{2\pi}{V}\sum_{\vt n \neq \vt 0} \frac{e^{-\frac{k^2}{4\alpha^2}}}{k^2} \sum_{I=1}^n \sum_{J=1}^n \lambda_{IJ} \sum_{i \in I} \sum_{j \in J} q_i q_j e^{\im \tr{\vt k}({\vt r}_i - {\vt r}_j)}.
\end{equation*}

It is then convenient to define a type-specific structure factor $S_I(\vt k) = \sum_{i \in I} q_i e^{\im \tr{\vt k}{\vt r}_i}$, which depends solely on the charges and positions of atoms of type $I$. From such definition we can again split the summations over $i$ and $j$ to write
\begin{equation}
\label{eq:U_long_by_types_1}
U_\text{long} = \frac{2\pi}{V}\sum_{\vt n \neq \vt 0} \frac{e^{-\frac{k^2}{4\alpha^2}}}{k^2} \sum_{I=1}^n \sum_{J=1}^n \lambda_{IJ} S_I(\vt k) \overline{S_J}(\vt k).
\end{equation}

The fact that $\lambda_{IJ} = \lambda_{JI}$ for every type pair implies that all imaginary terms in the double above will cancel out, allowing us to simplify the equation as
\begin{equation}
\label{eq:U_long_by_types_dot}
U_\text{long} = \frac{2\pi}{V}\sum_{\vt n \neq \vt 0} \frac{e^{-\frac{k^2}{4\alpha^2}}}{k^2} \sum_{I=1}^n \sum_{J=1}^n \lambda_{IJ} [S_I(\vt k) \cdot S_J(\vt k)],
\end{equation}
where the dot product between two complex numbers $x_i = a_i + \im b_i$ and $x_j = a_j + \im b_j$ is a commutative operation defined so that $x_i \cdot x_j = a_i a_j + b_i b_j$. Finally, we define $\sigma_I(\vt k) = \sum_{J=1}^n \lambda_{IJ} S_J(\vt k)$ and rewrite Eq.~\ref{eq:U_long_by_types_dot} as
\begin{equation*}
U_\text{long} = \frac{2\pi}{V}\sum_{\vt n \neq \vt 0} \frac{e^{-\frac{k^2}{4\alpha^2}}}{k^2} \sum_{I=1}^n S_I(\vt k) \cdot \sigma_I(\vt k).
\end{equation*}

Differentiating with respect to the position $\vt r_i$ of an atom $i \in I$, we have
\begin{equation*}
\frac{\partial U_\text{long}}{\partial \vt r_i} = \frac{2\pi}{V}\sum_{\vt n \neq \vt 0} \frac{e^{-\frac{k^2}{4\alpha^2}}}{k^2} \sum_{J=1}^n \left[ S_J(\vt k) \cdot \frac{\partial \sigma_J}{\partial \vt r_i} + \sigma_J(\vt k) \cdot \frac{\partial S_J}{\partial \vt r_i}\right],
\end{equation*}
where a trivial scalar$\cdot$vector generalization of the dot product has been employed. We then note that $\partial S_J / \partial \vt r_i = \vt 0$ for all $J \neq I$, which makes $\partial \sigma_J / \partial \vt r_i = \lambda_{IJ} {\partial S_I}/{\partial \vt r_i}$ and, therefore,
\begin{align*}
\frac{\partial U_\text{long}}{\partial \vt r_i} &= \frac{2\pi}{V}\sum_{\vt n \neq \vt 0} \frac{e^{-\frac{k^2}{4\alpha^2}}}{k^2} \left[ \sigma_I(\vt k) \cdot \frac{\partial S_I}{\partial \vt r_i} + \sum_{J=1}^n S_J(\vt k) \cdot \left( \lambda_{IJ} \frac{\partial S_I}{\partial \vt r_i} \right) \right] = \\
&= \frac{4\pi}{V}\sum_{\vt n \neq \vt 0} \frac{e^{-\frac{k^2}{4\alpha^2}}}{k^2} \left[ \sigma_I(\vt k) \cdot \frac{\partial S_I}{\partial \vt r_i} \right].
\end{align*}

From the definition of $S(\vt k)$, we obtain $\frac{\partial S_I}{\partial \vt r_i} = \im q_i e^{\im \tr{\vt k}{\vt r}_i} \vt k$. It is thus convenient to employ the anti-commutative cross product $x_i \times x_j = (\im x_i) \cdot x_j = a_ib_j- b_ia_j$. In this way, we have
\begin{equation*}
{\vt F}_i = -\frac{\partial U_\text{long}}{\partial \vt r_i} = \frac{4\pi}{V} q_i \sum_{\vt n \neq \vt 0} \frac{e^{-\frac{k^2}{4\alpha^2}}}{k^2} [ \sigma_I(\vt k) \times e^{\im \tr{\vt k}{\vt r}_i} ]{\vt k}.
\end{equation*}

\section{DRAFT}

For computational efficiency:
\begin{equation*}
U_\text{long} = \frac{2\pi}{V}\sum_{\vt n \neq \vt 0} \frac{e^{-\frac{k^2}{4\alpha^2}}}{k^2} \sum_I \bigg\{\lambda_{II} \|S_I(\vt k)\|^2 + 2 S_I(\vt k) \cdot \sum_{J > I} \lambda_{IJ} S_J(\vt k)\bigg\}.
\end{equation*}

Note that a position $\vt r_i$ only affects the structure factor of a type $I$ if $i \in I$. In order to facilitate differentiation, we extract from $U_\text{long}$ all terms that involve a given type $I$, which is
\begin{equation*}
U_I = \frac{2\pi}{V}\sum_{\vt n \neq \vt 0} \frac{e^{-\frac{k^2}{4\alpha^2}}}{k^2} \bigg\{\lambda_{II} \|S_I(\vt k)\|^2 + 2 \bigg(\sum_{J \neq I} \lambda_{IJ} S_J(\vt k)\bigg) \cdot S_I(\vt k) \bigg\},
\end{equation*}

Taking the derivative with respect to $\vt r_i$:
\begin{equation*}
\frac{\partial U_I}{\partial \vt r_i} = \frac{2\pi}{V}\sum_{\vt n \neq \vt 0} \frac{e^{-\frac{k^2}{4\alpha^2}}}{k^2} \bigg\{2 \lambda_{II} S_I(\vt k) + 2 \bigg(\sum_{J \neq I} \lambda_{IJ} S_J(\vt k)\bigg) \bigg\}  \cdot \frac{\partial S_I}{\partial \vt r_i},
\end{equation*}

Therefore,
\begin{equation*}
\frac{\partial U_I}{\partial \vt r_i} = \frac{4\pi}{V} \sum_{\vt n \neq \vt 0} \frac{e^{-\frac{k^2}{4\alpha^2}}}{k^2} \bigg[ \sum_J \lambda_{IJ} S_J(\vt k) \bigg] \cdot \frac{\partial S_I}{\partial \vt r_i},
\end{equation*}

\begin{equation*}
\frac{\partial S_I}{\partial \vt r_i} = \im q_i e^{\im \tr{\vt k}{\vt r}_i}{\vt k}
\end{equation*}

Defining:
\begin{equation*}
\sigma_I(\vt k) = \sum_J \lambda_{IJ} S_J(\vt k)
\end{equation*}

\begin{equation*}
\frac{\partial U_I}{\partial \vt r_i} = \frac{4\pi}{V} \sum_{\vt n \neq \vt 0} \frac{e^{-\frac{k^2}{4\alpha^2}}}{k^2} \sigma_I(\vt k) \cdot \frac{\partial S_I}{\partial \vt r_i},
\end{equation*}

\begin{equation*}
\frac{\partial U_I}{\partial \vt r_i} = \frac{4\pi q_i}{V} \sum_{\vt n \neq \vt 0} \frac{e^{-\frac{k^2}{4\alpha^2}}}{k^2} [\sigma_I(\vt k) \times e^{\im \tr{\vt k}{\vt r}_i}]{\vt k},
\end{equation*}

\section{DRAFT}


\begin{equation*}
U_\text{long} = \frac{2\pi}{V}\sum_{\vt n \neq \vt 0} \frac{e^{-\frac{k^2}{4\alpha^2}}}{k^2} \sum_I S_I(\vt k) \cdot \Bigg[
\lambda_{II}S_I(\vt k) + \sum_{J \neq I} \lambda_{IJ} S_J(\vt k)\Bigg],
\end{equation*}



Force:
\begin{equation*}
\frac{\partial}{\partial \vt r_n}(S_I \cdot S_J) = S_I \cdot \frac{\partial S_J}{\partial \vt r_n} + S_J \cdot \frac{\partial S_I}{\partial \vt r_n},
\end{equation*}

\begin{equation*}
\frac{\partial S_I}{\partial \vt r_n} = \begin{cases}
\im q_n e^{\im \tr{\vt k}{\vt r}_n}{\vt k} & \text{if} \; n \in I \\
0 & \text{if} \; n \notin I
\end{cases}
\end{equation*}

\begin{equation*}
e^{\im \tr{\vt k}{\vt r}_i} = \cos(\tr{\vt k}{\vt r}_i) + \im \sin(\tr{\vt k}{\vt r}_i)
\end{equation*}



\begin{equation}
\label{eq:U_long_by_types_1}
U_\text{long} = \frac{2\pi}{V}\sum_{\vt n \neq \vt 0} \frac{e^{-\frac{k^2}{4\alpha^2}}}{k^2} \Bigg\{ \sum_I \lambda_{II} \|S_I(\vt k)\|^2 + 2 \sum_{J>I} \lambda_{IJ} [S_I(\vt k) \cdot S_J(\vt k)] \Bigg\},
\end{equation}
where a dot product between two complex numbers is defined so that
\begin{equation*}
(a_1 + \im b_1)\cdot(a_2 + \im b_2) = a_1 a_2 + b_1 b_2.
\end{equation*}

The derivative of a structure factor $S_I(\vt k)$
\begin{equation*}
\frac{\partial S_I}{\partial \vt r_i} = \im q_i e^{\im \tr{\vt k}{\vt r}_i} \vt k
\end{equation*}


In order the further develop equation above, let us define $a_j = q_j \cos(\tr{\vt g}{\vt r}_j)$ and $b_j = q_j \sin(\tr{\vt g}{\vt r}_j)$, so that $q_j e^{i \tr{\vt g}{\vt r}_j} = a_j + ib_j$. Thus,
\begin{equation*}
S_J(\vt g) = \sum_{j \in J} (a_j + i b_j) = A_J + i B_J,
\end{equation*}
where $A_J = \sum_{j \in J} a_j$ and $B_J = \sum_{j \in J} b_j$. The complex conjugate of $S_J(\vt g)$ is $\overline{S_J(\vt g)} = A_J - i B_J$, which allows us to write
\begin{equation*}
S_J(\vt g)\overline{S_K(\vt g)} = A_J A_K + i(B_J A_K - A_J B_K) + B_J B_K,
\end{equation*}

It is clear that the imaginary term above vanishes if $K = J$. Moreover, $\lambda_{JK}$ and $\lambda_{KJ}$ being equal implies that all imaginary terms cancel out in the double sum of Eq.~\ref{eq:U_long_by_types_1}, which thus becomes
\begin{equation*}
U_\text{long} = \frac{2\pi}{V}\sum_{\vt n \neq \vt 0} \frac{e^{-\frac{g^2}{4\alpha^2}}}{g^2} \sum_J \sum_K \lambda_{JK} (A_J A_K + B_J B_K).
\end{equation*}

This can also be rewritten as
\begin{equation*}
U_\text{long} = \frac{2\pi}{V}\sum_{\vt n \neq \vt 0} \frac{e^{-\frac{g^2}{4\alpha^2}}}{g^2} \sum_J \left[\lambda_{JJ} |S_J|^2 + 2 \sum_{K>J} \lambda_{JK} S_J \cdot S_K\right],
\end{equation*}
where $|S_J|^2 = A_J^2 + B_J^2$


The force exerted on an atom $j$ is
\begin{equation*}
F_j^\text{long} = -\frac{\partial U_\text{long}}{\partial \vt r_j} = \frac{2\pi}{V}\sum_{\vt n \neq \vt 0} \frac{e^{-\frac{g^2}{4\alpha^2}}}{g^2} \sum_J \sum_K \lambda_{JK} (A_J A_K + B_J B_K).
\end{equation*}

The needed derivatives are
\begin{align*}
\frac{\partial A}{\partial \vt r_j} &= \frac{\partial a_j}{\partial \vt r_j} = -q_j \sin(\tr{\vt k}{\vt r}_j){\vt k} = -b_j{\vt k} \\
\frac{\partial B}{\partial \vt r_j} &= \frac{\partial b_j}{\partial \vt r_j} = q_j \cos(\tr{\vt k}{\vt r}_j){\vt k} = a_j{\vt k}
\end{align*}

Therefore,
\begin{equation*}
F_j = \frac{4\pi}{V}\sum_{\vt n \neq \vt 0} \frac{e^{-\frac{k^2}{4\alpha^2}}}{k^2} \left( A b_j - B a_j \right){\vt k}.
\end{equation*}

\section{DRAFT}
\begin{equation*}
(a_i + \im b_i)\cdot(-b_j + \im a_j) = -a_i b_j + b_i a_j.
\end{equation*}
\begin{equation*}
(-b_i + \im a_i)\cdot(a_j + \im b_j) = -b_i a_j + a_i b_j.
\end{equation*}


where $S_J = \sum_{j \in J} q_j e^{i \tr{\vt g}{\vt r}_j}$ is the structure factor resulting from the atoms of type $J$ alone. If we define $a_j = q_j \cos(\tr{\vt g}{\vt r}_j)$ and $b_j = q_j \sin(\tr{\vt g}{\vt r}_j)$, so that $q_j e^{i \tr{\vt g}{\vt r}_j} = a_j + ib_j$, then we have
\begin{equation*}
S_J(\vt g) = \sum_{j \in J} (a_j + i b_j) = A_J + i B_J,
\end{equation*}
where $A_J = \sum_{j \in J} a_j$ and $B_J = \sum_{j \in J} b_j$. Thus, $|S(\vt g)|^2 = A^2 + B^2$.

The Fourier part of the system energy is:
\begin{equation*}
U = \frac{2\pi}{V}\sum_{\vt n \neq \vt 0} \frac{e^{-\frac{k^2}{4\alpha^2}}}{k^2} |S(\vt k)|^2,
\end{equation*}
where $\vt n$ is a vector in $\mathbb Z^3$ (integer lattice), $\vt k = 2\pi \vt L^{-1}{\vt n}$ is a reciprocal space vector, and $k = \|\vt k\|$ is the norm of $\vt k$. The function $S(\vt k)$ is the structure factor, defined as
\begin{equation*}
S(\vt k) = \sum_{j=1}^N q_j e^{i\tr{\vt k}{\vt r}_j}.
\end{equation*}

If we define $a_j = q_j \cos(\tr{\vt k}{\vt r}_j)$ and $b_j = q_j \sin(\tr{\vt k}{\vt r}_j)$, then we have
\begin{equation*}
S(\vt k) = \sum_{j=1}^N \left[a_j + i b_j\right] = A + i B,
\end{equation*}
where $A = \sum_j a_j$ and $B = \sum_{j=1}^N b_j$, so that $|S(\vt k)|^2 = A^2 + B^2$. The force over an atom $j$ is
\begin{equation*}
F_j = -\frac{\partial U}{\partial \vt r_j} = -\frac{4\pi}{V}\sum_{\vt n \neq \vt 0} \frac{e^{-\frac{k^2}{4\alpha^2}}}{k^2} \left( A \frac{\partial A}{\partial \vt r_j} + B \frac{\partial B}{\partial \vt r_j} \right).
\end{equation*}

The needed derivatives are
\begin{align*}
\frac{\partial A}{\partial \vt r_j} &= \frac{\partial a_j}{\partial \vt r_j} = -q_j \sin(\tr{\vt k}{\vt r}_j){\vt k} = -b_j{\vt k} \\
\frac{\partial B}{\partial \vt r_j} &= \frac{\partial b_j}{\partial \vt r_j} = q_j \cos(\tr{\vt k}{\vt r}_j){\vt k} = a_j{\vt k}
\end{align*}

Therefore,
\begin{equation*}
F_j = \frac{4\pi}{V}\sum_{\vt n \neq \vt 0} \frac{e^{-\frac{k^2}{4\alpha^2}}}{k^2} \left( A b_j - B a_j \right){\vt k}.
\end{equation*}

We define an operation between two complex number as
\begin{equation*}
(a_1 + i b_1)\circ(a_2 + i b_2) = a_1 b_2 - b_1 a_2.
\end{equation*}

In this manner,
\begin{equation*}
F_j = \frac{4\pi}{V}\sum_{\vt n \neq \vt 0} \frac{e^{-\frac{k^2}{4\alpha^2}}}{k^2} \left[ S(\vt k) \circ e^{i\tr{\vt k}{\vt r}_j} \right]{\vt k}.
\end{equation*}

\end{document}
