\documentclass[11pt]{article}

\usepackage{amsmath}
\usepackage{amssymb}

\newcommand{\mt}[1]{\boldsymbol{\mathbf{#1}}}           % matrix symbol
\newcommand{\vt}[1]{\boldsymbol{\mathbf{#1}}}           % vector symbol
\newcommand{\tr}[1]{#1^\text{t}}                        % transposition
\newcommand{\diff}[2]{\frac{\partial #1}{\partial #2}}  % partial derivative

\begin{document}

\title{EmDee: A Molecular Dynamics Laboratory}

\author{Charlles R. A. Abreu}
%\email{abreu@eq.ufrj.br}
%\affiliation{Chemical Engineering Department, Escola de Quimica, Universidade Federal do Rio de Janeiro, Rio de Janeiro, RJ 21941-909, Brazil}

\date{\today}

\maketitle

\section{Projections and Vector Differentiation}

Consider a column vector $\vt x \in \mathcal{R}^n$, whose norm is $x = \sqrt{\tr{\vt x}{\vt x}}$. A unit vector in the direction of $\vt x$ is given by $\vt{\hat x} = {\vt x}/x$. When premultiplied by an arbitrary vector in $\mathcal{R}^n$, the matrix obtained by the product $\vt{\hat x}\tr{\vt{\hat x}}$ provides the component of such vector in the direction of $\vt x$, while $\mt I - \vt{\hat x}\tr{\vt{\hat x}}$ provides its projection onto the plane orthogonal to $\vt x$. Such as any projection matrix, both $\vt{\hat x}\tr{\vt{\hat x}}$ and $\mt I - \vt{\hat x}\tr{\vt{\hat x}}$ are idempotent.

Here we deal with column vectors and employ the so-called Jacobian formulation (also known as numerator layout) for derivatives. In this case, differentiation of a vector-valued function $\vt y \in \mathcal{R}^m$ with respect to a vector variable $\vt x \in \mathcal{R}^n$ results in a matrix $\mt J \in \mathcal{R}^{m \times n}$ so that $J_{ij} = \partial y_i / \partial x_j$. With this formulation, the total differential of a generic function $\vt z = {\vt z}(\vt x,\vt y)$ is given by 
\[
d{\vt z} = \diff{\vt z}{\vt x} d{\vt x} + \diff{\vt z}{\vt y} d{\vt y},
\]
which is valid even if $\vt x$, $\vt y$ or $\vt z$ is a scalar. From this expression, many useful differentiation rules involving vectors can be derived. For instance,
\begin{equation}
\label{eq:prod_rule_1}
d(\tr{\vt x}{\vt y}) = \tr{\vt x}d{\vt y} + \tr{\vt y}d{\vt x}
\end{equation}
and
\begin{equation}
\label{eq:prod_rule_2}
d({\vt x}y) = {\vt x}dy + y d{\vt x}.
\end{equation}

A special case of Eq.~\ref{eq:prod_rule_1} is $d(\tr{\vt x}{\vt x}) = 2\tr{\vt x}d{\vt x}$, which can be used to obtain the differential $dx = d(\sqrt{\tr{\vt x}{\vt x}}) = \frac{1}{2}(\tr{\vt x}\vt x)^{-1/2}d(\tr{\vt x}\vt x)$, whose final form is
\begin{equation}
\label{eq:norm_rule}
dx = \frac{\tr{\vt x}d{\vt x}}{x}.
\end{equation}

Similarly, we have $d({\vt x}x^{-1}) = -{\vt x}x^{-2}dx + x^{-1}d\vt x$ as a special case of Eq.~\ref{eq:prod_rule_2}. This can be rewritten as
\begin{equation}
\label{eq:unit_vector_rule}
d\hat{\vt x} = \frac{\mt I - \hat{\vt x}\tr{\hat{\vt x}}}{x} d\vt x,
\end{equation}
which shows us that a differential change in a vector $\vt x$ causes an orthogonal differential change in the unit vector it generates. Next, we consider the differential form of a projection like $\hat{\vt x} \tr{\hat{\vt x}} \vt y$. By applying Eq.~\ref{eq:prod_rule_2}, followed by Eq.~\ref{eq:prod_rule_1}, it turns out that
\begin{equation}
\label{eq:projection_rule}
d(\hat{\vt x} \tr{\hat{\vt x}} \vt y) = \hat{\vt x}\tr{\hat{\vt x}} d\vt y + (\hat{\vt x}\tr{\vt y} + \tr{\hat{\vt x}} \vt y \mt I) d\hat{\vt x}.
\end{equation}

Finally, consider a pair of matrix-valued functions with vector arguments, $A$ and $B$, defined so that $A(\vt x)\vt y = \vt z = B(\vt y)\vt x$. In this case, it follows from Eq.~\ref{eq:prod_rule_1} that
\begin{equation}
\label{eq:interchange_rule}
d\vt z = A(\vt x)d\vt y + B(\vt y)d\vt x.
\end{equation}

\section{Torsional Forces}

In order to calculate the torsional forces, we begin by defining an orthonormal, right-handed basis composed by the vectors
\begin{subequations}
\begin{align}
\hat{\vt z} &= \dfrac{{\vt r}_{kj}}{r_{kj}}, \\
\hat{\vt x} &= \dfrac{\vt x}{x}, \, \text{where} \, \vt x = (\mt I - \hat{\vt z}\tr{\hat{\vt z}}){\vt r}_{ij}, \, \text{and} \label{eq:x_definition} \\
\hat{\vt y} &= \hat{\vt z} \times \hat{\vt x}.
\end{align}
\end{subequations}

This serves as an internal coordinate system for the dihedral. Since it is an orthonormal basis for $\mathcal{R}^3$, it follows that $\tr{\hat{\vt x}}\hat{\vt y} = \tr{\hat{\vt x}}\hat{\vt z} = \tr{\hat{\vt y}}\hat{\vt z} = 0$ and that $\hat{\vt x}\tr{\hat{\vt x}} + \hat{\vt y}\tr{\hat{\vt y}} + \hat{\vt z}\tr{\hat{\vt z}} = {\mt I}$. The dihedral angle can be calculated using the arctangent function with two arguments,
\begin{subequations}
\begin{equation}
\phi = \arctan(b,a),
\end{equation}
where
\begin{align}
a &= \tr{\hat{\vt x}}{\vt r}_{lk} \quad \text{and}\\
b &= \tr{\hat{\vt y}}{\vt r}_{lk}.
\end{align}
\end{subequations}

The differential of $\phi$ is given by
\[
d\phi = \frac{a db - b da}{a^2 + b^2} = \frac{\tr{(a\hat{\vt y}-b\hat{\vt x})}d{\vt r}_{lk} + \tr{{\vt r}_{lk}}(a d\hat{\vt y} - b d\hat{\vt x})}{a^2 + b^2}.
\]

Note that $\hat{\vt y}$ is obtained from $\hat{\vt z}$ and $\hat{\vt x}$ by a cross product, which can be expressed in matrix form as $\hat{\vt y} = S(\hat{\vt z})\hat{\vt x}$, where $S(\hat{\vt z})$ is an antisymmetric matrix. Since $\hat{\vt z} \times \hat{\vt x} = - \hat{\vt x} \times \hat{\vt z}$, then $S(\hat{\vt z})\hat{\vt x} = - S(\hat{\vt x})\hat{\vt z}$ and, from Eq.~\ref{eq:interchange_rule}, it follows that $d\hat{\vt y} = S(\hat{\vt z})d\hat{\vt x} - S(\hat{\vt x})d\hat{\vt z}$. Therefore,
\[
\tr{{\vt r}_{lk}}d\hat{\vt y} = \tr{{\vt r}_{lk}}S(\hat{\vt z})d\hat{\vt x} - \tr{{\vt r}_{lk}}S(\hat{\vt x})d\hat{\vt z} = \tr{({\vt r}_{lk} \times \hat{\vt z})}d\hat{\vt x} - \tr{({\vt r}_{lk} \times \hat{\vt x})}d\hat{\vt z}.
\]

Substituting this result into $d\phi$ then yields
\[
(a^2 + b^2)d\phi = \tr{\left(a\hat{\vt y} - b\hat{\vt x}\right)}d{\vt r}_{lk}
+ \tr{\left(a{\vt r}_{lk} \times \hat{\vt z} - b{\vt r}_{lk}\right)}d\hat{\vt x}
- \tr{\left(a{\vt r}_{lk} \times \hat{\vt x}\right)}d\hat{\vt z}.
\]

From Eqs.~\ref{eq:unit_vector_rule}, \ref{eq:projection_rule}, and \ref{eq:x_definition}, it follows that
\[
d\hat{\vt x} = \frac{\mt I - \hat{\vt x}\tr{\hat{\vt x}}}{x} \left[ (\mt I - \hat{\vt z}\tr{\hat{\vt z}})d{\vt r}_{ij} - (\hat{\vt z}\tr{{\vt r}_{ij}} + \tr{\hat{\vt z}} {\vt r}_{ij} \mt I) d\hat{\vt z} \right]
\]

Using the identities $(\mt I - \hat{\vt x}\tr{\hat{\vt x}})(\mt I - \hat{\vt z}\tr{\hat{\vt z}}) = \hat{\vt y}\tr{\hat{\vt y}}$ and $(\mt I - \hat{\vt x}\tr{\hat{\vt x}})\hat{\vt z} = \hat{\vt z}$, and the fact that $\tr{\hat{\vt z}} {\vt r}_{ij}$ is a scalar, we have
\[
d\hat{\vt x} = \frac{\hat{\vt y}\tr{\hat{\vt y}}}{x}d{\vt r}_{ij} - \frac{\hat{\vt z}\tr{{\vt r}_{ij}} + \tr{\hat{\vt z}} {\vt r}_{ij} (\mt I - \hat{\vt x}\tr{\hat{\vt x}}) }{x}d\hat{\vt z}
\]

Now, for simplification, we define
\begin{subequations}
\begin{align}
&\vt u = \frac{a ({\vt r}_{lk} \times \hat{\vt z}) - b {\vt r}_{lk}}{x}, \\
&\vt v = \frac{a ({\vt r}_{lk} \times \hat{\vt x}) + (\tr{\hat{\vt z}}\vt u) {\vt r}_{ij}}{r_{kj}}, \; \text{and} \\
&\vt w = {\vt v} + \frac{\tr{\hat{\vt z}}{\vt r}_{ij}}{r_{kj}} {\vt u}
\end{align}
\end{subequations}

Next, substituting $\vt u$ and $\vt v$ and observing that $d\hat{\vt z} = (\mt I - \hat{\vt z}\tr{\hat{\vt z}}) d{\vt r}_{kj}/r_{kj}$, we obtain
\[
(a^2 + b^2)d\phi = \tr{\left(a\hat{\vt y} - b\hat{\vt x}\right)}d{\vt r}_{lk}
+ \tr{\vt u}\hat{\vt y}\tr{\hat{\vt y}}d{\vt r}_{ij} - \frac{ r_{kj} \tr{\vt v} + \tr{\hat{\vt z}} {\vt r}_{ij} \tr{\vt u}(\mt I - \hat{\vt x}\tr{\hat{\vt x}}) }{r_{kj}} (\mt I - \hat{\vt z}\tr{\hat{\vt z}}) d{\vt r}_{kj}.
\]

To conclude, the identities $(\mt I - \hat{\vt x}\tr{\hat{\vt x}})(\mt I - \hat{\vt z}\tr{\hat{\vt z}}) = \hat{\vt y}\tr{\hat{\vt y}}$ and $\mt I - \hat{\vt z}\tr{\hat{\vt z}} = \hat{\vt x}\tr{\hat{\vt x}} + \hat{\vt y}\tr{\hat{\vt y}}$, together with the definition of $\vt w$, yield
\[
(a^2 + b^2)d\phi = \tr{\left(a\hat{\vt y} - b\hat{\vt x}\right)}d{\vt r}_{lk}
+ \tr{\vt u}\hat{\vt y}\tr{\hat{\vt y}}d{\vt r}_{ij} - (\tr{\vt v}\hat{\vt x}\tr{\hat{\vt x}} + \tr{\vt w}\hat{\vt y}\tr{\hat{\vt y}}) d{\vt r}_{kj}.
\]

For a given torsional potential $U(\phi)$, the force exerted on an atom $m$ is obtained by
\[
\vt{F}_m = -\diff{U}{\phi} \tr{ \left( \diff{\phi}{\vt{r}_m} \right) }
\]

Therefore,
\begin{subequations}
\begin{align}
&\vt{F}_i = -\diff{U}{\phi} \frac{(\tr{\vt u}\hat{\vt y})\hat{\vt y}}{a^2 + b^2} \\
&\vt{F}_j = -\diff{U}{\phi} \frac{ (\tr{\vt v}\hat{\vt x})\hat{\vt x} + (\tr{\vt w}\hat{\vt y} - \tr{\vt u}\hat{\vt y}) \hat{\vt y}}{a^2 + b^2} \\
&\vt{F}_k = -\diff{U}{\phi} \frac{ (b - \tr{\vt v}\hat{\vt x})\hat{\vt x} - (a + \tr{\vt w}\hat{\vt y}) \hat{\vt y}}{a^2 + b^2} \\
&\vt{F}_l = -\diff{U}{\phi} \frac{a\hat{\vt y} - b\hat{\vt x}}{a^2 + b^2}
\end{align}
\end{subequations}

\end{document}
